\documentclass[a4paper, 11pt]{article}
\usepackage{times}
\usepackage[utf8]{inputenc}
\usepackage[czech]{babel}
\usepackage[left=2cm, text={17cm, 24cm}, top=3cm]{geometry}
\usepackage{multirow}
\usepackage{graphics}
\usepackage[ruled, czech, noline, linesnumbered, longend]{algorithm2e}
\usepackage{pdflscape}
\usepackage{graphics}


\begin{document}
\begin{titlepage}
    \begin{center}
        \textsc{{\Huge Vysoké učení technické v~Brně}\\[0.4em]
                {\huge Fakulta informačních technologií}}\\
        \vspace{\stretch{0.382}}
        {\LARGE Typografie a publikování\,--\,3. projekt}\\[0.3em]
        {\Huge Tabulky a obrázky}
        \vspace{\stretch{0.618}}
    \end{center}
{\Large \today \hfill Ján Homola}
\end{titlepage}

\section{Úvodní strana}
Název práce umístěte do zlatého řezu a nezapomeňte uvést \uv{dnešní} datum a vaše jméno a příjmení.

\section{Tabulky}
Pro sázení tabulek můžeme použıt buď prostředí \texttt{ tabbing } nebo prostředí \texttt{ tabular}.

\subsection{Prostředí \texttt{ tabbing}}
Při použití \texttt{ tabbing } vypadá tabulka následovně:

\begin{tabbing}
    Vodní molouny \quad \= Cena \quad \= Množství \= \kill
    \textbf{Ovoce} \> \textbf{Cena} \> \textbf{Množství} \\
    Jablka \> 25,90 \> 3 kg \\
    Hrušky \> 27,40 \> 2,5 kg \\
    Vodní molouny \> 35,-- \> 1 kus
\end{tabbing}
\bigskip

\noindent Toto prostředí se dá také použít pro sázení algoritmů, ovšem vhodnější je použít
prostředí \texttt{ algorithm } nebo \texttt{ algorithm2e } (viz sekce \ref{Alg}).

\subsection{Prostředí \texttt{ tabular}}
Další možností, jak vytvořit tabulku, je použít prostředí tabular. Tabulky pak
budou vypadat takto\footnote{Kdyby byl problem s \texttt{ cline}, zkuste se podívat třeba sem:
http://www.abclinuxu.cz/tex/poradna/show/325037}:
\bigskip

\begin{table}[h]
    \centering
    \catcode`\-=12
    \begin{tabular}{| l | r | r |} \hline
        & \multicolumn{2}{ c |}{\textbf{Cena}}\\ \cline{2-3}
        \textbf{Měna} & \textbf{nákup} & \textbf{predaj}\\ \hline
        EUR & 24,775 & 25,943\\
        GBP & 29,394 & 30,492\\
        USD & 22,423 & 23,661\\ \hline
    \end{tabular}
    \caption{Tabulka kurzů k dnešnímu dni}
    \label{tab:tab1}
\end{table}
\bigskip

\begin{table}[h]
    \centering
    \catcode`\-=12
    \begin{tabular}{| c | c |} \hline
         $A$ & $\neg A$ \\ \hline
         \textbf{P} & N \\
         \textbf{O} & O \\
         \textbf{X} & X \\
         \textbf{N} & p \\ \hline
    \end{tabular}
    \begin{tabular}{| c | c | c | c | c | c |} \hline
         \multicolumn{2}{| c |}{\multirow{2}{*}{$A \wedge B$}} & \multicolumn{4}{| c |}{$B$}\\ \cline{3-6}
         \multicolumn{2}{| c |}{} & \textbf{P} & \textbf{O} & \textbf{X} & \textbf{N}  \\ \hline
         \multirow{4}{*}{$A$} & \textbf{P} & P & O & X & N \\ \cline{2-6}
         & \textbf{O} & O & O & N & N \\ \cline{2-6}
         & \textbf{X} & X & N & X & N \\ \cline{2-6}
         & \textbf{N} & N & N & N & N \\ \hline
    \end{tabular}
    \begin{tabular}{| c | c | c | c | c | c |} \hline
         \multicolumn{2}{| c |}{\multirow{2}{*}{$A \vee B$}} & \multicolumn{4}{| c |}{$B$}\\ \cline{3-6}
         \multicolumn{2}{| c |}{} & \textbf{P} & \textbf{O} & \textbf{X} & \textbf{N}  \\ \hline
         \multirow{4}{*}{$A$} & \textbf{P} & P & P & P & P \\ \cline{2-6}
         & \textbf{O} & P & O & P & O \\ \cline{2-6}
         & \textbf{X} & P & P & X & X \\ \cline{2-6}
         & \textbf{N} & P & O & X & N \\ \hline
    \end{tabular}
    \begin{tabular}{| c | c | c | c | c | c |} \hline
         \multicolumn{2}{| c |}{\multirow{2}{*}{$A \vee B$}} & \multicolumn{4}{| c |}{$B$}\\ \cline{3-6}
         \multicolumn{2}{| c |}{} & \textbf{P} & \textbf{O} & \textbf{X} & \textbf{N}  \\ \hline
         \multirow{4}{*}{$A$} & \textbf{P} & P & O & X & N \\ \cline{2-6}
         & \textbf{O} & P & O & P & O \\ \cline{2-6}
         & \textbf{X} & P & P & X & X \\ \cline{2-6}
         & \textbf{N} & P & P & P & P \\ \hline
    \end{tabular}
    \caption{Protože Kleeneho trojhodnotová logika už je \uv{zastaralá}, uvádíme si zde příklad čtyřhodnotové logiky}
    \label{tab:tab2}
\end{table}
\bigskip
\pagebreak

\section{Algoritmy}
\label{Alg}
Pokud budeme chtít vysázet algoritmus, můžeme použít prostředí \texttt{ algorithm\footnote{Pro nápovědu, jak zacházet s prostředím \texttt{ algorithm}, můžeme zkusit tuhle stránku:\\
http://ftp.cstug.cz/pub/tex/CTAN/macros/latex/contrib/algorithms/algorithms.pdf.} } nebo \texttt{ algorithm2e\footnote{Pro \texttt{ algorithm2e } zase tuhle: http://ftp.cstug.cz/pub/tex/CTAN/macros/latex/contrib/algorithm2e/doc/algorithm2e.pdf.}}.
Příklad použití prostředí \texttt{ algorithm2e } viz Algoritmus 1.
\bigskip
\begin{algorithm}
\caption{\textsc{FastSLAM}}
\label{fastSLAM_alg}
\SetAlgoNlRelativeSize{-1}
\SetNlSkip{-1em}
\SetInd{1em}{0em}
\SetNlSty{}{}{:}
\KwIn{$ (X_{t-1}, u_t, z_t)$}
\KwOut{$ X_t$}
\BlankLine
\Indp
\Indpp
$\overline{X_t} = X_t = 0$\\
\For{$k = 1$ \emph{to} $M$}
{
    $x^{[k]} = $ \emph{sample\_motion\_model} $(u_t, x_{t-1}^{[k]})$\\
    $\omega_{t}^{[k]} = $ \emph{measurement\_model} $(z_{t},x_{t}^{[k]},m_{t-1})$\\
    $m_{t}^{[k]} = $ \emph{updated\_occupancy\_grid} $(z_{t},x_{t}^{[k]},m_{t-1}^{[k]})$\\
    $\overline{X_t} = \overline{X_t} + \langle x_{x}^{[m]},\omega_{t}^{[m]} \rangle$
}
\For{$k = 1$ \emph{to} $M$}
{
    draw $i$ with probability $\approx \omega_{t}^{[i]}$\\
    add $\langle x_{x}^{[k]},m_{t}^{[k]}\rangle$ to $X_t$
}
\Return{$ X_t$}
\end{algorithm}

\section{Obrázky}
Do našich článků můžeme samozřejmě vkládat obrázky. Pokud je obrázkem fotografie, můžeme klidně použít
bitmapový soubor. Pokud by to ale mělo být nějaké schéma nebo něco podobného, je dobrým zvykem takovýto
obrázek vytvořit vektorově.

\begin{figure}[h]
    \centering
    \scalebox{0.4}
    {
        \includegraphics{etiopan.eps}
        \reflectbox{
        \includegraphics{etiopan.eps}
        }
    }
    \caption{Malý Etiopánek a jeho bratříček}
    \label{fig:Etiopanek}
\end{figure}
\pagebreak

\noindent Rozdíl mezi vektorovým \dots

\begin{figure}[h]
    \centering
    \scalebox{0.4}
    {
        \includegraphics{oniisan.eps}
    }
    \caption{Vektorový obrázek}
    \label{fig:vektorovy}
\end{figure}

\noindent \dots a bitmapovým obrázkem

\begin{figure}[h]
    \centering
    \scalebox{0.6}
    {
        \includegraphics{oniisan2.eps}
    }
    \caption{Bitmapový obrázek}
    \label{fig:bitmapovy}
\end{figure}

\noindent se projeví například při zvětšení.

Odkazy (nejen ty) na obrázky \ref{fig:Etiopanek}, \ref{fig:vektorovy} a \ref{fig:bitmapovy}, na tabulky \ref{tab:tab1} a \ref{tab:tab2} a také na algoritmus \ref{fastSLAM_alg} jsou udělány pomocí křížových odkazů. 
Pak je ovšem potřeba zdrojový soubor přeložit dvakrát.

Vektorové obrázky lze vytvořit i přímo v \LaTeX u, například pomocí prostředí \texttt{ picture}.

\pagebreak
% vykreslenie domu
\begin{landscape}
\begin{figure}
    \centering
    \setlength{\unitlength}{1mm}
    \begin{picture}(200, 100)
        \linethickness{1pt}
        \put(0, 0){\framebox(200, 100){}}
        
        % slnko
        \put(170, 80){\circle{13}}
        % hruba ciara
        \linethickness{1.2mm}
        \put(5, 15){\line(1, 0){190}}
        % tenke ciary
        \linethickness{1pt}
        \put(35, 15){\line(0, 1){13}}
        \put(35, 28){\line(1, 0){38}}
        \put(73, 28){\line(3, -1){38}}
        
        \put(88, 23){\line(1, 0){96}}
        \put(184, 23){\line(0, -1){8}}
        
        \put(182, 23){\line(0, 1){14}}
        \put(182, 37){\line(-1, 0){105}}
        \put(77, 37){\line(0, -1){10}}
        
        \put(184, 39){\line(-1, 0){140}}
        \put(44, 39){\line(1, -1){11}}
        \put(184, 44){\line(-1, 0){140}}
        \put(184, 44){\line(0, -1){5}}
        \put(44, 44){\line(0, -1){5}}
        
        \put(70, 44){\line(0, 1){8}}
        \put(70, 52){\line(1, 0){55}}
        \put(125, 52){\line(0, -1){8}}
        
        \put(125, 46){\line(1, 0){50}}
        \put(175, 46){\line(0, -1){2}}
        
        \put(70, 48){\line(-1, 0){45}}
        \put(25, 48){\line(0, -1){33}}
        
    \end{picture}
    \caption{Vektorový obrázek moderního bydlení vhodného pro 21. století. (Buď to vytvořte stejný obrázek, anebo nakreslete pomocí \texttt{ picture } váš vlastní domov.)}
    \label{fig:hause}
\end{figure}
\end{landscape}


\end{document}
