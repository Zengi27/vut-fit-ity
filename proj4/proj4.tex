\documentclass[a4paper, 11pt]{article}
\usepackage{times}
\usepackage[utf8]{inputenc}
\usepackage[czech]{babel}
\usepackage[left=2cm, text={17cm, 24cm}, top=3cm]{geometry}
\usepackage[unicode]{hyperref}
\bibliographystyle{czplain}

\begin{document}
\begin{titlepage}
    \begin{center}
        \textsc{{\Huge Vysoké učení technické v~Brně}\\[0.4em]
                {\huge Fakulta informačních technologií}}\\
        \vspace{\stretch{0.382}}
        {\LARGE Typografie a publikování\,--\,4. projekt}\\[0.3em]
        {\Huge Bibliografické citácie}
        \vspace{\stretch{0.618}}
    \end{center}
{\Large \today \hfill Ján Homola}
\end{titlepage}


\section{Typografia}
Typografie je dnes základnou súčasťou našich životov, je vyvrcholením storočí vývoja, pretože písmena, ktoré tvoria písanie slovo sa vyvinuli a vykryštalizovali do bežne používaných písmen \cite{typography_fundamentals}.

Typografiu môžme chápať ako druh umenia alebo zručnosť v~oblasti usporiadavania písmen a znakov.
V~podstate typografia zahŕňa to ako sú usporiadané jednotlivé slova a písmena, a ako sú zložené vo
vzťahu k~sebe navźajom a k~danému miestu v~kompozícií \cite{harkins2011basics}. 

Existuje niekoľko nepísaných pravidiel, ktoré by mal dodržiavať každý dobrý dizajnér. Konkrétne príklady je možné nájsť na tejto stránke \cite{typho_rules}.


\section{\LaTeX}
História \LaTeX u začína s~programom, ktorý sa nazýva \TeX a vznikol v~roku 1978 \cite{tex_history}.
Nejedná sa o~sádzací program, ale skôr o~nádstavbu pre DTP systému \TeX. Pomocou tohto balíka je používanie \TeX u jednoduchšie. Takéto \LaTeX\ dokumenty je potrebné zkompilovať, pretože neprebiehajú interaktívne ako u~iných textových editorov \cite{diplomka_latex}. 

Ako silné stránky \LaTeX u sa považuje napríklad sádzanie matematických výrazov, obrovská presnosť a taktiež možno vytvorenia vlastných makier \cite{martinek_latex}.
\LaTeX\ obsahuje aj nástroj na automatizáciu použitej literatúry, tento nástroj sa nazýva Bib\TeX \cite{bib}. 

\subsection{Online editory \LaTeX}
Online editory sú veĺkým prínosom, umožňujú nám pristúpiť k~naším dokumentom, bez toho aby sme museli
sedieť pri našom počítači. Taktieť nás odbremenie od samotnéj inštalácie editoru
\cite{diplomka_online_editory}. Medzi online editory patria napríklad \cite{latex_example_editors}:

\begin{itemize}
    \item Papeeria
    \item Overleaf
    \item Authorea
\end{itemize}

\section{Predspracovanie textovych dát}
Predspracovanie textových dát pre klasifikáciu pomocou strojového učenia spočíva v~nasledujúcich krokoch \cite{classification}:

\begin{itemize}
    \item tokenizácia
    \item odstránenie stop slov
    \item prevod textu na veľké písmena
    \item odstránenie šumu
    \item stematizácia a lematizácia
\end{itemize}

\newpage
\renewcommand{\refname}{Literatúra}
\bibliography{proj4}

\end{document}
