\documentclass[a4paper, 11pt, twocolumn]{article}
\usepackage{times}
\usepackage[utf8]{inputenc}
\usepackage[czech]{babel}
\usepackage[left=1.5cm, text={18cm, 25cm}, top=2.5cm]{geometry}
\usepackage[IL2]{fontenc}
\usepackage{hyperref}
\usepackage{xcolor}
\usepackage{amsthm}
\usepackage{amsmath}
\usepackage{amsfonts}

\theoremstyle{definition}
\newtheorem{definition}{Definice}
\newtheorem{lemma}{Věta}

\begin{document}
\begin{titlepage}
\begin{center}
    \Huge
    \textsc{{\Huge Vysoké učení technické v~Brně}\\
            {\huge Fakulta informačních technologií}}\\
    \vspace{\stretch{0.382}}
    {\LARGE Typografie a publikování -- 2. projekt\\
    Sazba dokumentů a matematických výrazů}
    \vspace{\stretch{0.618}}
\end{center}
{\Large 2022 \hfill Ján Homola (xhomol27)}
\end{titlepage}
    


\section*{Úvod}
V~této úloze si vyzkoušíme sazbu titulní strany, matematických vzorců, prostředí a dalších textových struktur obvyklých pro technicky zaměřené texty (například rovnice \eqref{equation_2} nebo Definice \ref{def_2} na straně \pageref{def_2}). Pro vytvoření těchto odkazů používáme příkazy \verb|\label|, \verb|\ref| a \verb|\pageref|. \par
Na titulní straně je využito sázení nadpisu podle optického středu s~využitím zlatého řezu. Tento postup byl probírán na přednášce. Dále je na titulní straně použito odřádkování se zadanou relativní velikostí 0,4\,em a 0,3\,em.

\section{Matematický text}
Nejprve se podíváme na sázení matematických symbolů a výrazů v~plynulém textu včetně sazby definic a vět s~využitím balíku \texttt{amsthm}. Rovněž použijeme poznámku pod čarou s~použitím příkazu \verb|\footnote|. Někdy je vhodné použít konstrukci \verb|${}$| nebo \verb|\mbox{}|, která říká, že (matematický) text nemá být zalomen. 

\begin{definition}
\label{def_1}
Nedeterministický Turingův stroj (\emph{NTS) je šestice tvaru} $M = (Q, \Sigma, \Gamma, \delta, q_0, q_F)$, \emph{kde:}
\begin{itemize}
    \item \emph{Q je konečná množina} vnitřních (řídicích) stavů, 
    \item \emph{$\Sigma$ je konečná množina symbolů nazývaná} vstupní abeceda, $\Delta \notin \Sigma,$
    \item \emph{$\Gamma$ je konečná množina symbolů, $\Sigma \subset \Gamma, \Delta \in \Gamma,$ nazývaná} pásková abeceda,
    \item  $\delta:(Q \,\backslash \,\{q_F\}) \times \Gamma \rightarrow 2^{Q \times(\Gamma \cup\{L, R\})},$ \emph{kde} $L, R \notin \Gamma$, \emph{je parciálni} prechodová funkce, \emph{a}
    \item $q_0 \in Q$ \emph{je} počáteční stav \emph{a $q_F \in Q$ je} koncový stav.
\end{itemize}
\end{definition}

Symbol $\Delta$ značí tzv. \emph{blank} (prázdný symbol), který se vyskytuje na místech pásky, která nebyla ještě použita.\par
\emph{Konfigurace pásky} se skládá z~nekonečného řetězce, který reprezentuje obsah pásky, a pozice hlavy na tomto řetězci. Jedná se o~prvek množiny $\{\gamma \Delta^{\omega} \mid \gamma \in \Gamma^{*}\} \times \mathbb{N}$.\footnote{Pro libovolnou abecedu $\Sigma$ je $\Sigma^{\omega}$ množina všech \emph{nekonečných} řetězců nad $\Sigma$, tj. nekonečných posloupností symbolů ze $\Sigma$.} 
\emph{Konfiguraci pásky} obvykle zapisujeme jako $\Delta xyz\underline{z}x\Delta\ldots$ (podtržení značí pozici hlavy).
\emph{Konfigurace stroje} je pak dána stavem řízení a konfigurací pásky. Formálně se jedná o~prvek množiny $Q \times \{\gamma \Delta^{\omega} \mid \gamma \in \Gamma^{*}\} \times \mathbb{N}$.

\subsection{Podsekce obsahující definici a větu}
\begin{definition}
\label{def_2}
Řetězec $w$ nad abecedou $\Sigma$ je přijat NTS~$M$,   \emph{jestliže} $M$ \emph{při aktivaci z~počáteční konfigurace pásky} $\underline{\Delta}w\Delta\ldots$ \emph{a počátečního stavu} $q_0$ \emph{může zastavit přechodem do koncového stavu} $q_F$, \emph{tj.} $(q_0,\Delta w \Delta^{\omega},0)\underset{M}{\overset{*}{\vdash}} (q_F,\gamma,n)$ \emph{pro nějaké} $\gamma \in \Gamma^{*}$ \emph{a} $n \in \mathbb{N}$.\par
\emph{Množinu} $L(M) = \{w \mid w $\emph{ je přijat NTS} $M\} \subseteq\Sigma^{*}$ \emph{nazýváme} jazyk přijímaný NTS $M$.

\end{definition}

Nyní si vyzkoušíme sazbu vět a důkazů opět s~použitím balíku \texttt{amsthm}.

\begin{lemma}
\emph{Třída jazyků, které jsou přijímány NTS, odpovídá} rekurzivně vyčíslitelným jazykům.
\end{lemma}

\section{Rovnice}
Složitější matematické formulace sázíme mimo plynulý text. Lze umístit několik výrazů na jeden řádek, ale pak je třeba tyto vhodně oddělit, například příkazem \verb|\quad|.

$$
x^{2}-\sqrt[4]{y_{1} * y_{2}^{3}} \quad x>y_{1} \geq y_{2} \quad z_{z_{z}} \neq \alpha_{1}^{\alpha_{2}^{\alpha_{3}}}
$$

V~rovnici \eqref{equation_1} jsou využity tři typy závorek s~různou explicitně definovanou velikostí.

\begin{eqnarray}
    \label{equation_1} x & = &\bigg\{a \oplus\Big[b \cdot\big(c \ominus d\big)\Big]\bigg\}^{4 / 2} \\
    \label{equation_2} y & = &\lim _{\beta \rightarrow \infty} \frac{\tan ^{2} \beta-\sin ^{3} \beta}{\frac{1}{\frac{1}{\log _{42} x} +\frac{1}{2}}}
\end{eqnarray}

V~této větě vidíme, jak vypadá implicitní vysázení limity $\lim_{n \rightarrow \infty} f(n)$ v~normálním odstavci textu. Podobně je to i s~dalšími symboly jako $\bigcup_{N \in \mathcal{M}} N$ či $\sum_{j=0}^{n} x_{j}^{2}$. 
S~vynucením méně úsporné sazby příkazem \verb|\limits| budou vzorce vysázeny v~podobě $\lim\limits_{n \rightarrow \infty} f(n)$ a $\sum\limits_{j=0}^{n} x_{j}^{2}$. 

\section{Matice}
Pro sázení matic se velmi často používá prostředí \texttt{array} a závorky (\verb|\left|, \verb|\right|). 

$$
\mathbf{A}=\left|
\begin{array}{cccc}
    a_{11} & a_{12} & \ldots & a_{1 n} \\
    a_{21} & a_{22} & \ldots & a_{2 n} \\
    \vdots & \vdots & \ddots & \vdots \\
    a_{m 1} & a_{m 2} & \ldots & a_{m n}
\end{array}
\right|=\left|
\begin{array}{cc}
    t & u \\
    v & w
\end{array}\right|=t w-u v
$$

Prostředí \texttt{array} lze úspěšně využít i jinde.

$$
\binom{n}{k}
=\left\{
\begin{array}{cl}
    \frac{n !}{k !(n-k) !} & \text { pro } 0 \leq k \leq n \\
    0 & \text { pro } k>n \text { nebo } k<0
\end{array}\right.
$$

\end{document}
